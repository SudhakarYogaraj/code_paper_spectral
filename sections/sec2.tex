\section{Spectral method for the solution of the cell problem}
\label{sec:spectral_method_for_the_solution_of_the_cell_problem}

From the properties of the modified Hermite polynomials that we saw in the
previous section, the solution of the cell problem can be expanded in the following way
$$\phi(x,y) \,=\, \sum_{|\alpha|>0} \left \langle f, \hermite H_\alpha \right
\rangle_\rho \, \hermite H_\alpha.$$  
The series can be truncated, and the approximate solution obtained will be a
polynomial of finite degree. Clearly, assuming that we only keep the terms up
to degree $d$, this solution will be equal to $\pi^d \phi (x,\cdot)$, the
$L^2(\real^n, \rho)$ projection of $\phi$ on the space of polynomials of degree
lower or equal to $d$. By proposition \ref{proposition: approx Hermite} The
accuracy of the approximation will depend on the smoothness and decay
properties of $\phi(x,\,\cdot\,)$, which can be related to the smoothness of
$f(x,\,\cdot\,)$ via the following lemma.

\begin{lemma}[Regularity of the solution]
    Assume that $g \in \wsobolev{s}{\real^n}{\rho}$,
    with $s\in \nat$. Then there exists a unique solution of the problem
    $$
    \left\{ 
        \begin{aligned}
            &\op L_0 \, u = g   \\
            &u \in \wsobolev{1}{\real^n}{\rho}  
        \end{aligned} 
    \right .
    \quad \text{subject to the condition} \quad \int_{\real^n} u \, \dd \rho = 0.
    $$
    In addition, $u \in H^{s+2}(\real^n, \rho)$ and the following estimate holds
    $$
        \norm{u}_{\wsobolev {s+2} {\real^n} {\rho}} \,\leq\, C \, \norm{g}_{\wsobolev{s}{\real^n}{\rho}}.
    $$
    % , and if $g \in  \smooth{\real^n}$,
    % then $u \in \smooth{\real^n}$ also. 
    \label{lemma: regularity of the solution of the cell problem}
\end{lemma}
\iflong \begin{proof}
    Let us define the space 
    $$
        \Space H \,=\,\left\{u\,:\, u\,{\in}\,H^1(\real^n,\rho)\,\text{ and }\,\langle
            u,\,1\rangle_{\rho}\,=\, 0\right\}. 
    $$
    This is a Hilbert space for the inner product:
    $$
        \langle u,\,v\rangle_{\Space H} \,=\, \langle {\nabla}u,\, {\nabla}v\rangle_{\rho}.  
    $$
    The fact that this indeed defines an inner product follows from the
    Poincar\'e inequality in weighted Sobolev spaces (see
    \citep{arnold2007interpolation,beckner1989generalized}), which states that
    for any function $u\,\in\,\wsobolev{1}{\real^n}{\rho}$,
    $$
        \int_{\real^n} (u\,-\,\bar u)^2 \, \dd \rho\, \leq \,C\,\int_{\real^n} |\nabla
        u|^2 \,\dd \rho \quad \text{with}\quad \bar u \,=\, \int_{\real^n} u\,\dd \rho.
    $$
    A function $u\in \Space H$ is solution of the problem in the
    distributional sense if and only if
    $$
        a(u,v) = \frac{1}{2} \int_{\real^n}  \Sigma : \nabla u \nabla v \,
        \dd \rho \,=\, \int_{\real^n} g \,v\,\dd \rho \quad \forall v \in C^\infty_c(\real^n).
    $$
    By density of $\test{\real^n}$ in $\Space H$, this has to hold for any $v \in
    \Space H$. It can then be checked that the hypotheses of the Lax-Milgram
    theorem are met, which guarantees the existence and unicity of a solution 
    $u \in \Space H$. Using Poincar\'e inequality and the coercivity of
    $a(\cdot,\cdot)$, we have also that
    $$
        \norm{u}_{\wsobolev{1}{\real^n}{\rho}}^2 \,\leq\, C  \norm{u}_{\Space
            H}^2 \,\leq\, C\,a(u,u) \,=\, C\, \ip{g}{u}_{\rho} \,\leq\, C
        \norm{g}_{\wlp{2}{\real^n}{\rho}}
        \norm{u}_{\wsobolev{1}{\real^n}{\rho}}, 
    $$
    from which a stability estimate can be deduced. In particular, this
    justifies why the solution of the cell problem can be obtained using its
    expansion (\red{Insert reference here}).

    To prove that $u\in \wsobolev{s+2}{\real^n}{\rho}$, we will show that 
    $$
        u_n \,=\, \sum_{|\alpha| \leq n} \frac{1}{\mu_\alpha} \ip{g}{\hermite
            H_\alpha}_\rho\, \hermite H_\alpha
    $$
    is converges in $\wsobolev{s+2}{\real^n}{\rho}$.
    We can assume wlog that $\rho = g_{D_\infty}$, since this case can be
    recovered by rotation.  The sequence $D^\beta u_n$ is clearly convergent in
    $\wlp{2}{\real^n}{\rho}$ when $|\beta| \,\leq\, s$, because $D^\beta u \in
    \wlp{2}{\real^n}{\rho}$. Now assume that $\beta=\gamma + \delta$, with
    $|\gamma| = s$ and $|\delta| = 1$ or 2. Then by \cref{proposition: coeff
        Hermite multidimensional} and \cref{eq: recursion Hermite derivative},
    $$
        D^\beta u_n =D^\delta \left(\sum_{|\alpha|\leq n-s}\frac{1}{\mu_{\gamma+ \alpha}}\ip{D^\gamma g}{\hermite H_\alpha}\hermite  H_\alpha \right) \,=\, \sum_{|\alpha|\leq n-s}\frac{1}{\mu_{\gamma+ \alpha}}\ip{D^\gamma g}{\hermite H_\alpha}\hermite  D^\delta H_\alpha. 
    $$
    Using the fact that $\wlp{2}{\real^n}{ \rho}$ is a Hilbert space, it
    suffices to show that: 
    $$
        \sum_{\alpha\geq 0}\norm{\frac{D^\delta \hermite H_\alpha}{\mu_{\gamma+
                    \alpha}}}_{\rho}^2 \ip{D^\gamma g}{\hermite
            H_\alpha}_{\rho}^2 < \infty.   
    $$
    which follows by bounding the first term, and from the fact that $D^\gamma
    g \in \wsobolev{s}{\real^n}{\rho}$ by assumption. Note that, since the
    first term can be bounded by a constant independent of $\alpha$, the
    stability estimate follows
    $$
        \norm{u}_{\wsobolev{s+2}{\real^n}{\rho}} \,\leq\, C \norm{g}_{\wsobolev{s}{\real^n}{\rho}},
    $$
    which concludes the proof.
\end{proof} \fi
Using the approximate solution of the cell problem, we can calculate the
effective drift and diffusion coefficients of the simplified equation, using
\begin{equation}
    \begin{gathered}
        F^d (x) = \int_{\real^n} \left(\grad_x \phi_d\,\cdot\,f\,+\,\grad_y \phi_d \cdot h\right) \,\dd \rho \\[0.5cm]
        A(x) A^T(x) = \frac{1}{2}\left(A_0 + A_0^T\right) \quad \text{with}\quad A_0^d (x) = 2 \int_{\real^n}  f \otimes \phi_d \dd \rho
    \end{gathered}
    \label{eq: coefficients of approximate equation}
\end{equation}
Note that the matrix $A$ is not uniquely defined, which makes it impossible to
establish strong approximation properties of the numerical scheme under
investigation. To quantify the error that results from these approximate
formulae, we will make some regularity and boundedness assumptions.
\begin{assumption}
    The functions of the problem, seen as $x-$dependent functions taking values
    in a functional space, satisfy the following regularity assumption:
    \begin{itemize}
       \item  $f \in \cont{1}{\real^m; H^s(\real^n,\rho)} \cap \cont{\infty}{\real^m; \wlp{2}{\real^n}{\rho}}$.
       \item $h \in \cont{\infty}{\real^m; \wsobolev{1}{\real^n}{\rho}}$.
    \end{itemize}
    \label{assumption: regularity of the functions of the problem}
\end{assumption}
 These hypotheses imply the smoothness of the coefficients of the simplified
 equations, and they can be used to obtain error estimates. This is shown in
 the next two lemmas. 
\begin{lemma}
    Under \cref{assumption: regularity of the functions of the problem}, the
    exact and approximate coefficients of the simplified equation, defined by 
    \cref{eq: coefficients of approximate equation} are smooth functions of $x$.
    \begin{proof}
        We only prove the statement for the drift term. We start by showing
        that $\phi(x) \in \cont {\infty} {\real^m; \wsobolev{2}{\real^n}{\rho}}$.
        For a function $u(x,y)$ defined on $\real^m \times \real^n$, note
        $\Delta^h_j u(x)\,=\, (u(x+h\, e_j) - u(x))/h$, where the dependence on
        $y$ is omitted for notational economy. By assumption, $\Delta^h_j
        f(x)$ converges in $\wsobolev{s}{\real^n}{\rho}$ as $h \to 0$ for all
        $x \in \real^m$. On the other hand, $\Delta^h_j \phi$ satisfies:
        $$ 
            \op L_0 \, \Delta^h_j \phi(x) \,=\, \Delta^h_j f(x) \quad \text{with the centering condition}.
        $$
        By continuity of the solution operator, $\Delta^h_j \phi$ converges in
        $\wsobolev{s+2}{\real^n}{\rho} $, and its limit $\partial_{x_j} \phi$ satisfies:
        $$ \op L_0 \, (\partial_{x_j} \phi) \,=\, \partial_{x_j} f \quad
        \text{with the centering condition}.$$ Repeating this argument leads to
        the desired conclusion. We have
        \begin{align*}
            F_i(x) &\,=\, \sum^{m}_{j=1}  \int_{\real^n} \left(\partial_{x_j}\phi_i
                \, f_j\,+\, \phi_i\, \partial_{y_j}^* h_j\right) \,\dd \rho(y) \\
            &\,=\, \sum^{m}_{j=1}
            \left(\ip{\partial_{x_j}\phi_i(x)}{f_j(x)}_{\rho}
                \,+\,\ip{\phi_i(x)}{\partial_{y_j}^*h_j(x)}_{\rho}\right) 
        \end{align*}
        Since the operator $\partial_{y_j}^*$ is continuous from
        $\wsobolev{1}{\real^n}{\rho}$ to $\wlp{2}{\real^n}{\rho}$, all the
        terms that appear in inner products belong to $\cont{\infty}{\real^m;
            \wlp{2}{\real^n}{\rho}}$, from which the conclusion is clear. 

        A similar reasoning can be applied for the simplified equation. The only difference 
        is that we have to show the additional fact that $\pi_d \phi \in \cont{\infty}{\real^m;
            \wlp{2}{\real^m}{\rho}}$. It follows from $\pi_d \Delta^h_j \phi =
        \Delta^h_j \pi_d \phi$ that $\pi_d (\partial_{x_j}\phi) =
        \partial_{x_j} (\pi_d \phi)$, by passing to the limit $h \to 0$ in
        $\wlp{2}{\real^n}{\rho}$, and so $\partial_{x_j} (\pi_d \phi) \in \cont{0}{\real^m;
            \wsobolev{2}{\real^n}{\rho}}$. This can be applied to derivatives
        of higher orders to give the conclusion. 
    \end{proof}
\end{lemma}
The next lemma quantifies the precision of the approximation of the coefficients of the simplified equation. It is based on the polynomial approximation in the space $\wlp{2}{\real^n}{\rho}$, developed in \cref{sec:spectral_method_for_the_solution_of_the_cell_problem}. 
\begin{lemma}
Assuming that the functions of the problem satisfy \cref{assumption: regularity
    of the functions of the problem}, 
\begin{gather*}
    |F(x) \,-\, \hat F(x)| \, \vee \, |A(x) \,-\, \hat A(x)| \, \leq \,  B(R) \,(d+1)^{-s/2},
\end{gather*}
for all $|x| \leq R$.
\label{lemma: Approximation coefficients}
\begin{proof}
    From the expressions of $F$ and $F_d$, and from Cauchy-Schwarz inequality, we have:
    $$ \sum^{m}_{j=1} \int_{\real^n}  \left((\partial_{x_j}\hat \phi_i \,-\, \partial_{x_j}\phi)\, f_j\right) \dd \rho \,\leq\, \norm{f}_{\wlp{2}{\real^n}{ \rho} }\sum^{m}_{j=1}  \norm{\partial_{x_j}\hat \phi \,-\, \partial_{x_j}\phi}_{\wlp{2}{\real^n}{ \rho}}. \, $$
    We saw in the proof of the previous lemma that $\partial_{x_i} \phi_i(x)
    \in \wsobolev{s+2}{\real^n}{\rho}$ and $\partial_{x_i} (\pi_d \phi) \,=\,
    \pi_d (\partial_{x_i}\phi)$. Consequently, the error estimate
    \cref{proposition: approx Hermite} implies
    $$
        \norm{\partial_{x_j}\phi- \partial_{x_j}\hat \phi}_{\wlp{2}{\real^n}{\rho} } \,\leq\, C(n,s) \, (d+1)^{-s/2} \norm{\partial_{x_j}\phi}_{\wsobolev{s+2}{\real^n}{\rho}}.
    $$
    The other term that appears in the drift coefficient of the simplified
    equation can be estimated using a similar argument:
    \begin{align*}
        \sum^{n}_{j=1} \int_{\real^n} h_j(x,y) \, \partial_{y_j} \left(\phi_i(x,y)-\hat\phi_i(x,y)\right) \dd \rho(y) \,&=\,\sum^{n}_{j=1} \int_{\real^n} \partial^*_{y_j} h_j(x,y) \, (\phi_i(x,y)-\hat \phi_i(x,y)) \dd \rho(y) \,\\ &\leq\,C \norm{h(x)}_{\wsobolev{1}{\real^n}{\rho}} \norm{\phi_i(x) - \hat \phi_i(x)}_{\wlp{2}{\real^n}{ \rho}}.
    \end{align*}
    It thus follows that
    \begin{multline*}
        |F(x) - \hat F(x)| \,\leq\, C \,(d+1)^{-s/2}\left(\norm{f(x)}_{\wlp{2}{\real^n}{\rho}} \,+\, \norm{h(x)}_{\wsobolev{1}{\real^n}{\rho}} \right) \\ \left(\norm{\grad_x f(x)}_{\wsobolev{s+2}{\real^n}{\rho}} \,+\,\norm{f(x)}_{\wsobolev{s+2}{\real^n}{\rho}}\right).
    \end{multline*}
    By assumption, the terms between brackets are bounded on compact sets,
    which concludes the proof for the case of the drift coefficent A similar
    reasoning, applied to the diffusion term, gives
    $$
        |A_0(x) - \hat A_0(x)| \,\leq\, C \, (d+1)^{-s/2}\norm{f(x)}_{\wlp{2}{\real^n}{\rho}} \norm{f(x)}_{\wsobolev{s+2}{\real^n}{\rho}}.
    $$
    
\end{proof}
\end{lemma}
\begin{remark}
   If the assumptions are strengthened to  
   \begin{itemize}
       \item $f \in \contb{1}{\real^m;\wsobolev{s}{\real^n}{\rho}} \cap \cont{\infty}{\real^m; \wlp{2}{\real^n}{\rho}}$,
       \item $h \in \contb{\infty}{\real^m; \wsobolev{1}{\real^n}{\rho}}$,
   \end{itemize}
   the convergence is uniform on the whole domain.
\end{remark}
We now have enough tools to prove the weak convergence of the scheme when the
degree of polynomial approximation tends to $+\infty$. The next theorem
quantifies the distance between the solution of the simplified equation, and
the approximate solution obtained by our numerical scheme. Three sources of error are the following:
\begin{itemize}
    \item The time discretization. 
    \item The truncation error on the coefficients.
    \item Approximate numerical integration.
\end{itemize}
\begin{theorem}
    Under assumptions \red{to be defined}, the numercial scheme converges weakly.
\end{theorem}
\begin{proof}
    The proof is based on the backward Kolmogorov equation
    \cite{kloeden1992numerical} and \cite{weinan2005analysis}. It is well known
    that for any function $g \in \test{\real^n}$, the evolution of the
    expectation $\expect \left[g(X(T)| X(s) = x)\right] \,=: \, u(t,x)$ is governed by the equation:
    \begin{equation*}
        \pard{u}{t} \,+\,\op L\,u  \,=\,0,  \quad \text{with the final condition} \quad u(x,0) =
        g(x). 
    \end{equation*}
    where $\op L$ is the generator of the Markov process $X$, defined by:
    \begin{equation*}
            \op L \,=\,F(x) \dotx \grad_x \,+\,\frac{1}{2} A(x)\,:\,\grad_x \grad_x.
    \end{equation*}
    The error can be written in terms of the solution of this equation.
    \begin{equation}
        \begin{aligned}
            \Delta \,=\, \abs{ \expect\, \hat X_{n_T}\,-\,\expect\, X_T } &\,=\, \abs{ \expect u(T,\hat X_{n_T}) \,-\,u(0,x_0) } \\
            &\,=\, \abs{ \sum_{n=1}^{n_T-1}\expect \left( u(t_{n+1}, \hat X_{n+1}) \,-\, u(t_n, \hat X_n) \right)}.
        \end{aligned}
        \label{eq: weak estimate 1}
    \end{equation}
    Now, we note $X_t^{s,x}$ the process defined by:
    \begin{equation*}
        X_t^{s,x} \,=\, x \,+\, \int_s^t F(X_r^{s,x}) \dd r \,+\, \int_s^t A(X_r^{s,x})  \dd W_r.
    \end{equation*}
    Using Ito's formula for $Z_t \,=\, u(t,X_t^{s,x})$, and the fact that $u$
    is solution of the backward Kolmogorov equation
    \begin{equation*}
        \expect Z_t - \expect Z_s \,=\, \int_s^t \left\{\pardl{u}{t}(r, X_r^{s,x}) \,+\, \op Lu(r, X_r^{s,x}) \right\} \dd r \,=\, 0,
    \end{equation*}
    from which we conclude that $\expect \left[u(t_{n+1},
        X_{t_{n+1}}^{s,x})\right] = \expect \left[u(t_n,
        X_{t_n}^{s,x})\right]$. Using this in \cref{eq: weak estimate 1}, we have
    \begin{equation*}
        \begin{aligned}
            \Delta &\,=\, \abs{ \sum_{n=1}^{n_T-1}\expect \left( u(t_{n+1}, \hat
                X_{n+1}) \,-\, u(t_n, \hat X_n) \right)
            \,-\,\expect\left(u(t_{n+1}, X^{t_{n},\hat X_n}_{t_{n+1}})
                \,-\,u(t_n,X^{t_n,\hat X_n}_{t_n})\right) } \\
        & \,=\, \abs{ \sum_{n=1}^{n_T-1} \expect \left( u(t_{n+1}, \hat
                X_{n+1}) \,-\, u(t_{n+1}, X^{t_{n},\hat X_n}_{t_{n+1}}) \right) }.
        \end{aligned}
    \end{equation*}
    On the other hand, using previous estimates,
    \begin{equation*}
        \hat X_{n+1} \,-\, X^{t_n,\hat X^n}_{t_{n+1}} \,=\,  
    \end{equation*}
    


\end{proof}



\begin{assumption}
    For every $x \in \real^d$, we have $f(x,\cdot)\in H^n(\real^N, \rho)^d$, $\nabla_x f(x,\cdot) \in H^n(\real^N, \rho)^(d\times d)$, and $h(x,\cdot) \in L^2(\real^N, \rho)^N$. In addition, these functions satisfy the 
    boundedness property that for all $R>0$, there exists $B_R$ such that
    $$
    \|h(x,\cdot)\|_{\rho}\,{\vee}\,
    \|f(x,\cdot)\|_{{\rho},n}\,{\vee}\,\|\nabla_x f(x,\cdot)\|_{{\rho},n}
    \,{\leq}\, B_R \quad \text{for all} \quad |x| \,{\leq}\,R.
    $$
\end{assumption}
\begin{assumption}
    The exact drift and diffusion coefficients satisfy the following
    Lipschitz condition:
    $$ 
    \left|F(a)-F(b)\right|^2\,{\vee}\,\left|A(a)-A(b)\right|^2\, {\leq} \,C_R\, |a-b|^2
    $$
    for any $a,b$ such that $a,b\, {\leq} \,R$.
    \label{assumption Lipshitz}
\end{assumption}
\begin{assumption}
    There exists $p>2$ and a constant $K$ independent of $d$ such that the solutions of the equations 
    $$
    \mathrm dX=F(X)\,\mathrm dt\,+\,A(X)\,\mathrm dW_t \quad \quad X(0) \,=\,X_0
    $$
    and 
    $$
    \mathrm dX_d=F_d(X_d)\,\mathrm dt\,+\,A_d(X_d)\,\mathrm dW_t  \quad \quad X_d(0) \,=\,X_0 
    $$
    satisfy
    $$
    \mathbb E\left[\sup_{0 {\leq} t  {\leq} T} |X(t)|^p\right]\,{\vee}\,\mathbb
    E\left[\sup_{0 {\leq} t  {\leq} T} |X_d(t)|^p\right] \, {\leq} \,K
    $$
    \label{assumption: bounded moments}
\end{assumption} 
\begin{theorem}
    Under the three assumptions made above, the error between the approximate
    and exact solutions of the simplified equation satisfies
    \begin{multline*}
        \mathbb E \left[\sup_{0\, {\leq} \,t\, {\leq} \,T} |X(t) \,-\,X_d(t)|^2 \right]
        \, {\leq} \,4\,(T\,+\,4)\,D_R\,T\,d^{-n}\,\exp \left(4\,(T\,+\,4)\,C_R\, T\right)
        \\ \,+\, 2\,K\,
        \left(\frac{2^p\,{\delta}}{p}\,+\,\frac{p-2}{R^p\,p\,{\delta}^{\frac{2}{p-2}}}\right).
    \end{multline*}
    for any ${\delta}\,>\,0$ and $R\,>\,X_0$.
    \label{theorem: main converging coeff}
\end{theorem}
\begin{proof}
    We start by defining the following stopping times:
    $$
    {\tau}_R\,=\,\inf\left\{t\, {\geq} \,0\,:\,X(t)\, {\geq} \,R\right\}, \quad \quad
    {\tau}_R^d\,=\, \inf\left\{t\, {\geq} \,0\,:\,X_d(t)\, {\geq} \,R\right\}, \quad \quad
    {\theta}_R^d\,=\, {\tau}_R\,{\wedge}\,{\tau}_R^d.
    $$
    We also define the error function $e_d(t)\,=\,X(t)\,-\,X_d(t)$. This satisfies
    \begin{equation}
        \mathbb E \left[\sup_{0\, {\leq} \,t\, {\leq} \,T}\,|e_d(t)|^2 \right]\, = \,\mathbb E
        \left[\sup_{0\, {\leq} \,t\, {\leq} \,T} |e_d(t)|^2 \,\mathbf 1_{{\theta}_R^d\,>\,T} \right] \,+\,
        \mathbb E \left[\sup_{0\, {\leq} \,t\, {\leq} \,T} |e_d(t)|^2 \, \mathbf 1_{{\theta}_R^d\, {\leq} \,T} \right].
        \label{eq: theorem 1st}        
    \end{equation}
    To bound the second term of \eqref{eq: theorem 1st}, we use Young's
    inequality \eqref{eq: Young} with $r \,=\, p/2$:
    \begin{equation}
        \mathbb E \left[\sup_{0\, {\leq} \,t\, {\leq} \,T} |e_d(t)|^2 \, \mathbf 1_{{\theta}_R^d  {\leq} T} \right]\,
        \, {\leq} \,\frac{2\,{\delta}}{p}\,\mathbb E \left[\sup_{0\, {\leq} \,t\, {\leq} \,T} |e_d(t)|^p\right]
        \,+\,\frac{p-2}{p\,{\delta}^{\frac{2}{p-2}}}\,\mathbb P({\theta}_R^d\, {\leq} \,T).
        \label{eq:auxEq1}
    \end{equation}
    Using this formula and assumption \ref{assumption: bounded moments} above,
    the two terms of the right-hand side of the above inequality can be bounded:
    \begin{equation}
        \mathbb E \left[\sup_{0\, {\leq} \,t\, {\leq} \,T} |e_d(t)|^p\right]\,=\,\mathbb E
        \left[\sup_{0\, {\leq} \,t\, {\leq} \,T} |X(t)\,-\,X_d(t)|^p\right]
        \, {\leq} \,2^{p}\,K,
        \label{eq:auxIneq1}
    \end{equation}
    and 
    \begin{equation}
        \begin{aligned}
            \mathbb P({\theta}_R^d\, {\leq} \,T)\,& {\leq} \,\mathbb P({\tau}_R \, {\leq} \,T) \,+\,\mathbb P({\tau}_R^d\, {\leq} \,T) \\
            & {\leq} \,\mathbb E \left[\sup_{0\, {\leq} \,t\, {\leq} \,T} \frac{|X(t)|^p}{R^p}\right] \,+\,
            \mathbb E \left[\sup_{0\, {\leq} \,t\, {\leq} \,T} \frac{|X_d(t)|^p}{R^p}\right]
            \, {\leq} \, \frac{2\,K}{R^p}. 
        \end{aligned}
        \label{eq: auxIneq2}
    \end{equation}
    Using \eqref{eq:auxEq1}, \eqref{eq:auxIneq1}, and \eqref{eq: auxIneq2} in
    inequality \eqref{eq: theorem 1st}, we obtain: 
    \begin{multline}
        \mathbb E \left[\sup_{0\, {\leq} \,t\, {\leq} \,T} |e_d(t)|^2 \right] \, {\leq} \,\mathbb E
        \left[\sup_{0\, {\leq} \,t\, {\leq} \,T} |e_d(t)|^2\,\mathbf 1_{{\theta}_R^d >T} \right]\,+\,
        2\, K\, \left(\frac{2^p\,{\delta}}{p}\,+\,\frac{p-2}{R^p\,p\,{\delta}^{\frac{2}{p-2}}}\right).
        \label{eq: theorem 2nd}        
    \end{multline}
    To bound the first term, we will use assumptions \ref{assumption Lipshitz}
    and \ref{assumption: convergence coeffs} above. First, notice that 
    \begin{equation}
        \mathbb E \left[\sup_{0\, {\leq} \,t\, {\leq} \,T} |e_d(t)|^2 \, \mathbf 1_{{\theta}_R^d >T} \right]
        \, {\leq} \,\mathbb E \left[\sup_{0\, {\leq} \,t\, {\leq} \,T} |e_d(t\,{\wedge}\,{\theta}_R^d)|^2 \right].
        \label{eq:reformulation error term}
    \end{equation}
    Since $e_d(t)\,=\,X(t)\,-\,X_d(t)$, it satisfies the following equation
    $$
    e_d(t\,{\wedge}\,{\theta}_R^d) \,=\, {\int}_0^{t\,{\wedge}\,{\theta}_R^d} F(X(s))\,-\,F_d(X_d(s))\, ds
    \,+\, {\int}_0^{t\,{\wedge}\,{\theta}_R^d} A(X(s))\,-\,A_d(X_d(s))\,\mathrm dW_s.
    $$
    Using the inequality $(a+b)^2\, {\leq} \,2a^2\,+\,2b^2$ and Cauchy-Schwarz, we have
    \begin{multline}
        \mathbb E \left[ \sup_{0\, {\leq} \,t\, {\leq} \,{\tau}} |e_d(t\,{\wedge}\,{\theta}_R^d)|^2\right]\, {\leq} \,
        2\,T\,\mathbb E\,\left[ {\int}_0^{{\tau}\,{\wedge}\,{\theta}_R^d}
            |F(X(s))\,-\,F_d(X_d(s))|^2 \,\mathrm ds\right] \,+\, \\ 2\,\mathbb E \left[\sup_{0 {\leq} 
                t {\leq} {\tau}} \left| 
                {\int}_0^{t\,{\wedge}\,{\theta}_R^d} A(X(s))\,-\,A_d(X_d(s))\,\mathrm dW_s \right|^2\right].
        \label{eq:main eq global}
    \end{multline}
    The two terms on the right-hand side can be bounded. For the first term, we
    use the triangle inequality and the assumptions to deduce:
    \begin{equation}
        \begin{aligned}
            {\int}_0^{{\tau}\,{\wedge}\,{\theta}_R^d} & |F(X(s))\,-\,F_d(X_d(s))|^2\,\mathrm ds \\ 
            &\, {\leq} \,{\int}_0^{{\tau}\,{\wedge}\,{\theta}_R^d}2|F(X(s))\,-\,F(X_d(s))|^2\,+\,2|F(X_d(s)) \,-\, F_d(X_d(s))|^2\,\mathrm ds  \\
            &\, {\leq} \, 2\,D_R\,T\,d^{-n}\,+\,2\,C_R\,{\int}_0^{{\tau}\,{\wedge}\,{\theta}_R^d}|X(s)\,-\,X_d(s)|^2\,\mathrm ds \\
            &\, {\leq} \, 2\,D_R\,T\,d^{-n}\,+\,2\,C_R\,{\int}_0^{{\tau}}|X(s\,{\wedge}\, {\theta}_R^d)\,-\,X_d(s\,{\wedge}\, {\theta}_R^d)|^2\,ds
        \end{aligned}
    \end{equation}
    so
    \begin{multline}
        \mathbb E \left[ {\int}_0^{{\tau}\,{\wedge}\,{\theta}_R^d}|F(y(s))\,-\,F_d(y_d(s))|^2\,\mathrm ds\right] \\
        \, {\leq} \, 2\,D_R\,T\,d^{-n}\,+\,2\,C_R\,{\int}_0^{{\tau}}\mathbb
        E\left[\sup_{0\, {\leq} \,t\, {\leq} \,s}|e_d(t \,{\wedge}\, {\theta}_R^d)|^2\right]\,ds
        \label{eq:1st term main eq}
    \end{multline}
    The second term can be bounded in a similar manner after using Doob's
    inequality \eqref{eq:Doob's martingale} with $p\,=\,2$ and It\=o's formula:
    \begin{equation}
        \begin{aligned}
            & \mathbb E \left[\sup_{0\, {\leq} \,t {\leq} \,{\tau}} \left|{\int}_0^{t\,{\wedge}\,{\theta}_R^d}
                    A(y(s))\,-\,A_d(y_d(s)) \,\mathrm dW_s \right|^2\right] \\
            & \quad \quad \quad \, {\leq} \,\,4\,\mathbb E \left|{\int}_0^{{\tau}\,{\wedge}\,{\theta}_R^d}
                A\,(y(s))\,-\,A_d(y_d(s))\, \mathrm dW_s\right|^2 \\
            & \quad \quad \quad {\leq} \, 4 \,\mathbb E{\int}_0^{{\tau}\,{\wedge}\,{\theta}_R^d} |A(y(s))\,-\,A_d(y_d(s))|^2\,\mathrm ds \\
            & \quad \quad \quad {\leq} \,4\,\mathbb E{\int}_0^{{\tau}\,{\wedge}\,{\theta}_R^d}2|A(y(s))\,-\,A(y_d(s))|^2\,+\,2|A(y_d(s))
            \,-\, A_d(y_d(s))|^2\,\mathrm ds  \\
            & \quad \quad \quad  {\leq} \, 8\,D_R\,T \,d^{-n}\,+\,8\,C_R\,{\int}_0^{{\tau}}\mathbb
            E\left[\sup_{0\, {\leq} \,t\, {\leq} \,s}|e_d(t \,{\wedge}\, {\theta}_R^d)|^2\right]\,ds
        \end{aligned}
        \label{eq:2nd term main eq}
    \end{equation}
    Using \eqref{eq:1st term main eq} and \eqref{eq:2nd term main eq} in
    \eqref{eq:main eq global}, we obtain:
    \begin{multline*}
        \mathbb E \left[ \sup_{0\, {\leq} \,t\, {\leq} \,{\tau}} |e_d(t\,{\wedge}\,{\theta}_R^d)|^2\right]\, {\leq} \,\\
        4\,(T\,+\,4)\,\left(D_R\,T \,d^{-n}\,+\,C_R\,{\int}_0^{{\tau}}\mathbb
            E\left[\sup_{0\, {\leq} \,t\, {\leq} \,s}|e_d(t \,{\wedge}\, {\theta}_R^d)|^2\right]\,\mathrm ds\right).
    \end{multline*}
    which by Gronwall's inequality implies:
    \begin{equation}
        \mathbb E \left[ \sup_{0\, {\leq} \,t\, {\leq} \,{\tau}} |e_d(t\,{\wedge}\,{\theta}_R^d)|^2\right]\, {\leq} \,\\
        4\,(T\,+\,4)\,D_R\,T\,d^{-n}\,\exp \left(4\,(T\,+\,4)\,C_R\, {\tau}\right)
    \end{equation}
    Combining this inequality and \eqref{eq:reformulation error term} in \eqref{eq:
        theorem 2nd}, the following bound is obtained:
    \begin{multline}
        \mathbb E \left[\sup_{0\, {\leq} \,t\, {\leq} \,T} |e_d(t)|^2 \right]
        \, {\leq} \,4\,(T\,+\,4)\,D_R\,T\,d^{-n}\,\exp \left(4\,(T\,+\,4)\,C_R\, T\right)
        \,+\, \\ 2\,K\,
        \left(\frac{2^p\,{\delta}}{p}\,+\,\frac{p-2}{R^p\,p\,{\delta}^{\frac{2}{p-2}}}\right).
        \label{eq: theorem 2nd}        
    \end{multline}
\end{proof}
\noindent From this theorem, it can be deduced that the solution of the approximate
simplified equation converges to the exact solution when $d\,\to\,{\infty}$. For any ${\varepsilon}$ it
is possible to choose ${\delta}$ such that the second term is bounded by
${\varepsilon}/3$. Given this value of ${\delta}$, $R$ can be chosen so that the last
term is also bounded by ${\varepsilon}/3$. Finally, it is clear that the first term can also
be made arbitrarily small by increasing the value of $d$.  In the most general
cases, the previous statement does not provide any information about the rate
of convergence when $d\,\to\,{\infty}$. However, if the additional assumptions
that $C_R\, {\leq} \,C$ and  $D_R\, {\leq} \,D$ are satisfied, for constants
$C$ and $D$ independent of $R$, the error evolves as
$$
\mathbb E \left[\sup_{0\, {\leq} \,t\, {\leq} \,T} |e_d(t)|^2 \right]\,=\,\mathcal O(d^{-n})
$$
This fact can be seen by taking ${\delta}\,=\, d^{-n}$ and $R\,=\,d^{n/(p-2)}$.
