\section{Introduction}
\label{sec:introduction}
While the numerical methods dedicated to multiscale stochastic differential
equations (SDEs) have been studied for over a decade, the ones for multiscale
stochastic partial differential equations (SPDEs) have only recently attracted
the attention of the scientific community, although they arise in several
applications, such as climate modelling and crystal growth. It is well known
that classical methods used to integrate stochastic equations are very
inefficient for multiscale problems.  Explicit methods suffer from stringent
restrictions on the time step, rendering the numerical integration of the
system very expensive.  This is particularly relevant when the system of SDEs
results from the discretization of a stochastic partial differential equation
that includes diffusion. In this case, stability of the time integration
imposes that ${\delta}t < C({\varepsilon}\,{\delta}x)^{2}$, where
${\varepsilon}\,{\ll}\,1$ is the parameter that measures the scale separation.
The issue of stability can be solved by using implicit methods, but these
require the resolution of a system of algebraic equations at each time step,
which can be a very costly operation for large systems. Furthermore, it was
shown that the solutions obtained by those methods do not have the correct
invariant measure.

% Work done
In 2012, A. Abdulle and G.A.  Pavliotis showed that the HMM could be used in
combination with a spectral method to obtain appro\-ximate solutions of
stochastic partial differential equations with multiple scales. The
computational cost of the method they propose is independent of the scale
separation of the problem, which is a major advantage over classical methods.
Although the numerical method that A. Abdulle and G. Pavliotis designed works
very well, it was only tested for SPDEs with quadratic nonlinearities.
Furthermore, the analytical calculations they present become intractable in the
case of nonlinearities of higher degrees. 

In this paper, we extend the work presented in \cite{abdulle2012numerical} for
the numerical solution of stochastic partial differential equations with
multiple scales. In line with the analytical calculations of the paper, we
devoloped a semi-analytical method, based on the \emph{symbolic math toolbox}
of MATLAB, to find approximate expressions of the coefficients of the
simplified equations that governs the time evolution of the slow modes of the
SPDEs considered. In addition, the numerical method was extended to SPDEs with
arbitrary nonlinearities, in a systematic manner. The program written requires
minimal input from the user, and was implemented to be as flexible and
efficient as possible. We are interested in multiscale stochastic partial
differential equations of the following general type:
\begin{equation*}
    du\,=\, \frac{1}{{\varepsilon}^2}  \op A u \,dt \,+\, \frac 1
    {\varepsilon} \, F(u) \, dt\, + \,\frac{1}{\varepsilon}  Q\,dW, 
    \label{eq: basis spde} 
\end{equation*}
posed in a bounded domain of $\real^m$ with suitable boundary conditions.
In this expression, $\op A$ is a differential operator, assumed to be
non-positive and self-adjoint in a Hilbert space $\op H$, and with compact
resolvent. It is furthermore assumed that $\op A$ has a finite dimensional
kernel, noted $\mathcal N$, which causes the multiscale nature of the problem.
The term $W$ denotes a cylindrical Wiener process on $\mathcal H$, and $Q$ is
the covariance operator associated with the noise. It is assumed that $Q$ and
$\op A$ commute, and that the noise only acts on the orthogonal complement
$\mathcal N^{\perp}$.  The function $F({\cdot})$ is an arbitrary function
representing the nonlinearity. 

Given the assumption that the differential operator $\op A$ is compact and
self-adjoint, there exists an orthonormal basis of the Hilbert space $\mathcal
H$ consisting of eigenfunctions of $\op A$. With each of these
eigenfunctions is associated a real eigenvalue $-{\lambda}_k$ such that
$\op A\,e_{k} = -{\lambda}_{k}\,e_{k}$. Since the differential operator is
assumed to be non-positive, its eigenvalues are negative and so
${\lambda}_k\,\geq\,0$ for $k= 1, 2,{\dots}$.  It is supposed that the
eigenpairs are numbered in such a way that the $N$ first eigenfunctions are in
the kernel of the differential operator. Formally, the cylindrical Brownian
motion can be expanded in the basis as $W(t) \,=\, \sum^{\infty}_{i=1} \, e_i
\, w_i(t)$, where $\left\{w_i\right\}_{i=1}^{\infty}$ are independent Brownian
motion. The assumption that the covariance operator $Q$ commutes with the
differential operator $\op A$ means that this operator is given by $Q\,e_i
\,=\, q_i\, e_i$, while the assumption that the noise only acts on \,$\op
N^{\perp}$ implies that $q_i \,=\, 0$ for $i\,=\,1,\,2,\,{\dots}\, , \, N$.

Making precise sense of equation \eqref{eq: basis spde} requires a detailed study
of Gaussian random variables in infinite dimensional spaces and stochastic
partial differential equations. This will not be done here, but the reader can
refer to \citep{da2008stochastic,hairerphd,hairer2009introduction} for a
complete treatment of these subjects. Rather, we will see the solution as an
element of the space $\mathcal H$ that is determined via its projections on the
eigenfunctions $e_i$. As we will see in the next section, those projections
satisfy a system of stochastic differential equations, which are more familiar.
